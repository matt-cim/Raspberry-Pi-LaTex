\documentclass[12pt]{extarticle}

\author{Matthew Cimerola}
\title{Raspberry Pi 4 Manual}

\usepackage[utf8]{inputenc}
\usepackage{hyperref}
\hypersetup{
    colorlinks=true,
    }
\usepackage{graphicx}
\usepackage{pdfpages}
\usepackage{xcolor}


\begin{document}

\maketitle

\tableofcontents
\newpage
\section{About}
The Pi's we are using are "Raspberry Pi 4 Model B" with 2GB of RAM. \href{https://www.amazon.com/Flirc-Raspberry-Pi-Case-Silver/dp/B07WG4DW52/ref=sr_1_5?keywords=Raspberry+Pi+4+Case&qid=1658852531&sr=8-5}{Link to the case the Pi's use}. 
\subsection{Hardware Monitoring}
For the most graphically appeasing view of the general hardware.\\ 
(From Desktop) "Start" (Pi icon in the top left) \rightarrow "Accessories" \rightarrow "Task Manager" 

\begin{figure}[ht]
    \centering
    \includegraphics[scale=0.5]{task-manager.pdf}
    \caption{Note this is taken from a Pi with 4GB RAM}
\end{figure}
Run "top" via the \textbf{Terminal} for a view with more details including tasks and users. I added a CPU temperature gauge on the Pi's that you can see on the taskbar. This was added by $ \textbf{Right Clicking the Taskbar} \rightarrow \textbf{Add/Remove Panel items} \rightarrow \textbf{Add} \rightarrow \textbf{CPU Temperature Monitor}. $ From what I have read, a safe temperature resides under 80$^{\circ}$C. Thankfully, the case on the Pi's provide a built in heat sink and a thermal pad which is placed directly on top of the CPU.


\newpage
\section{Network Configuration}
\subsection{Wireless Connection}
After right clicking on the wifi icon and selecting "Wireless \& Wired Network Settings", you will want to mimic the settings to follow this:
\begin{figure}[ht]
    \centering
    \includegraphics[scale=1]{network config.pdf}
    \caption{Ensure you are in "wlan0" tab}
\end{figure}
\newline This is for a static ip configuration, which is the preferred config. and the "8.8.8.8" is just the primary DNS server for Google. One thing I noticed that was causing frequent problems with the ip address being randomly changed was the "Automatically configure empty options" box being checked even though the settings above were typed into the box. Please make sure this is \textbf{not checked}.
\subsection{Finding ip via Terminal}
Typing in the command \textbf{ifconfig} will give you the netmask, and the ip can be found next to "inet" under the "wlan0" section.

\newpage
\section{How to Connect}
\subsection{VNC}
VNC viewer is what we are currently using for remote connection into the Pi's, and I was able to use VNC viewer without making an account. \\

Steps for setting up VNC Viewer:
\begin{enumerate}
  \item Download \href{https://www.realvnc.com/en/connect/download/vnc/}{VNC Server for the Raspberry Pi OS}
  \item Take note of the ip under "Connectivity" and the catchphrase under "Security" (see Figure 3)
  \item From the computer you want to connect from, download \href{https://www.realvnc.com/en/connect/download/viewer/}{VNC Viewer for the OS of this machine} 
  \item From VNC Viewer, type in the ip address from step 2 in the search bar (Figure 4)
  \item You should be prompted to type in the password for the login to the Pi's, then should be connected
\end{enumerate}
\begin{figure}[ht]
    \centering
    \includegraphics[scale=1]{vnc server.pdf}
    \caption{}
\end{figure}
\newpage
\begin{figure}[ht]
    \centering
    \includegraphics[scale=1]{vnc bar.pdf}
    \caption{}
\end{figure}

\subsection{SSH}
This is more straight forward than VNC. \newline \newline For Mac:
\begin{enumerate}
  \item Open the Terminal
  \item Type in \textbf{ssh pi@INSERT\_IP\_ADDRESS}
  \item You will be prompted to type in the Pi's password, then should be all good
\end{enumerate}
For Windows:
\begin{enumerate}
  \item Open PuTTy
  \item Type in the Pi's ip address in the host name box
  \item Ensure ssh is the connection type
  \item \textbf{open}
  \item login with credentials ex: \textbf{pi@Pi-\#}
\end{enumerate}
\newline
\textbf{Note: }In order to reboot the Pi's when connected through SSH, you must put the keyword \textbf{sudo} before commands that require sudo privileges. An example would be \textbf{sudo reboot}. 

\newpage
\section{Auto Start}
This section will detail how the Pi's were set up so that on boot, certain web pages are opened automatically. This is done all through the Pi's terminal. \newline \newline
{\fontfamily{qcr}\selectfont
\# this is a comment \newline \newline
\# from the home directory "pi@Pi-\#:~\$" \newline
\# need to get to the configuration settings \newline
cd .config \\~\\
\# creating two new directories, error presents itself \newline \# if directories already exist \newline
sudo mkdir -p lxsession/LXDE-pi \\~\\
\# opening the built-in autostart folder \newline
sudo nano lxsession/LXDE-pi/autostart \\~\\
\# copy this inside the editor \newline
@lxpanel --profile LXDE-pi
@pcmanfm --desktop --profile LXDE-pi
#@xscreensaver -no-splash
@xset s off
@xset -dpms
@xset s noblank
point-rpi
@chromium-browser --start-fullscreen --start-maximized https://YOURURL.com/ \newline
\# note the YOURURL will be changed with what you would want \newlineo
\# on autostart. the "#@xscreensaver -no-splash" ensures \newline \# the pi does not sleep \\~\\
CTRL + x(saves), Y (yes), enter \\~\\
\# ensure it works \newline
sudo reboot
}
\\~\\
\textbf{Note: }The browsers with the KPI dashboards utilize \href{https://chrome.google.com/webstore/detail/easy-auto-refresh/aabcgdmkeabbnleenpncegpcngjpnjkc?hl=en}{Easy Auto Refresh}. All chrome extensions from what I have seen \emph{should} be supported by Chromium, the browser the Pi's use. I also installed "Midori", a lightweight browser that can be used.
\newpage
\section{A Note On Credentials}
The password for all the Pi's are the same and is the same password used for local admin accounts. The username for all of them is "pi", however, the local name that you will see in the terminal is "pi@Pi-NUMBER\_1\_THROUGH\_4
\end{document}
