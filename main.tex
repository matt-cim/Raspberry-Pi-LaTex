\documentclass[12pt]{extarticle}

\author{Matthew Cimerola}
\title{Raspberry Pi 4 Manual}

\usepackage[utf8]{inputenc}
\usepackage{hyperref}
\usepackage{graphicx}
\usepackage{pdfpages}


\begin{document}

\maketitle

\tableofcontents
\newpage
\section{About}
The Pi's we are using are "Raspberry Pi 4 Model B" with 2GB of RAM. \href{https://www.amazon.com/Flirc-Raspberry-Pi-Case-Silver/dp/B07WG4DW52/ref=sr_1_5?keywords=Raspberry+Pi+4+Case&qid=1658852531&sr=8-5}{Link to the case the Pis use}. 
\subsection{Hardware Monitoring}
For the most graphically appeasing view of the general hardware.\\ 
(From Desktop) "Start" (Pi icon in the top left) \rightarrow "Accessories" \rightarrow "Task Manager" 

\begin{figure}[ht]
    \centering
    \includegraphics[scale=0.5]{task-manager.pdf}
    \caption{Note this is taken from a Pi with 4GB RAM}
\end{figure}
Run "top" via the \textbf{Terminal} for a view with more details including tasks and users. I added a CPU temperature gauge on the Pi's that you can see on the taskbar. This was added by $ \textbf{Right Clicking the Taskbar} \rightarrow \textbf{Add/Remove Panel items} \rightarrow \textbf{Add} \rightarrow \textbf{CPU Temperature Monitor}. $ From what I have read, a safe temperature resides under 80$^{\circ}$C. Thankfully, the case on the Pi's provide a built in heat sink and a thermal pad which is placed directly on top of the CPU.


\newpage
\section{Network Configuration}
\subsection{example subsection for network config}
text for network configuration

\newpage
\section{How to Connect}
text for autostart


\newpage
\section{Auto Start}
text for autostart


\end{document}
